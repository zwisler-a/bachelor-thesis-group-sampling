\documentclass[../../thesis.tex]{subfiles}
\graphicspath{{./resources/} }
\begin{document}

\section{Sensitivity analysis}
\citeauthor{saltelli2004global} describes sensitivity analysis as 
"[...] the study of how the uncertainty in the output of a model
(numerical or otherwise) can be apportioned to different sources of uncertainty in the model input" \cite{saltelli2004global}.
To understand this definition, we need to understand, what a model is. A model, in this context, is a mathematical
description of a system and its rules. Such models could be weather models to predict future rainfall or financial models
to predict the stock market. By defining the input of the model, we can get a prediction from the model.
In this work, we view a configurable software system as a black-box model on which we can
perform sensitivity analysis. Our input variables are the features the systems have, and our output variable is the
measurable quality attribute.
Such models, no matter what they predict or are, can be highly complex, and it can be hard to understand the relationship
between the inputs and the outputs of the models.
% Sensitivity analysis helps in understanding the relation between the output and the input variables of a model, in searching for errors in a model and creating better models.
The degree to which an output variable is affected by the change in an input variable is called sensitivity.
There are many approaches to performing sensitivity analysis.
One common and simple approach to sensitivity analysis is the one-at-a-time (OAT) approach.
In this experimental design, one input variable is changed, while all other values are kept at a baseline value.
By observing the changes in the output variable, one can determine the influence an input variable has.
Sensitivity can then be described, for example, by linear regression or partial derivatives.

Performing sensitivity analysis on a configurable software with the OAT-design would mean,
enabling one feature at a time while all other features are disabled and observing the change in the quality attribute under analysis.
We assume here, that the software system has no constraints among the configuration options.
By fitting a regression model we can determine the sensitivity of the quality attribute to the features enabled.



\end{document}