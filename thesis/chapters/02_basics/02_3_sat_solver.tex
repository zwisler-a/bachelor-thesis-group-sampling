\documentclass[../../thesis.tex]{subfiles}
\graphicspath{{./resources/} }
\begin{document}



\section{SAT-Solver}
The satisfiability problem in propositional logic (SAT) is the problem of determining the existence of any solution,
that satisfies a given boolean formula. A boolean formula is called satisfiable, if we can assign the variables
in the formula true or false values in such a way, that the formula evaluates to true.
As an example, the formula $A \lor B$ is satisfiable. We can prove this by assigning $A=true$ and $B=false$.
The resulting formula would be $true \lor false$, which evaluates to true, meaning the formula is satisfiable.
If a formula is not satisfiable, the formula is called unsatisfiable. An example of this would be the
formula $A \land \lnot A$. No possible assignment of $A$ would result in the formula being evaluated to true.
% todo quellen
SAT-problems arise in many application domains and, most notably for this work, can be used in validating
configurations for SPLs.

SAT-Solvers are programs, which aim to solve the SAT-problem. Even though the SAT-problem is NP-complete \cite{cook1971complexity},
modern SAT-Solvers can often handle problems with hundreds of thousands of variables and millions of constraints \cite{ohrimenko2009propagation}.
They often use algorithms such as DPLL \cite{sinz2007visualizing} or variations of it at their core and their optimization is a large
field of research on its own.
Many free and open-source implementations of SAT-Solvers exist \cite{web:minisat, web:BatSat,web:SATurne}.
In this work, we use Z3 \cite{web:z3}, which is an SMT-solver. Satisfiability modulo theories (SMT) generalize the
SAT-problem to formulas involving real numbers and various data structures. The use of an SMT-solver,
instead of an SAT-Solver, allows us to define cost functions to optimize the solution of the SMT-solver.

In this work, we make use of the SMT-solver Z3 to generate valid configurations for a configurable software system.
Like \citet{henard2014bypassing} defined it, we define our feature model as a set of boolean features
with a set of constraints over them. A configuration is a set of features, where all features in that
set are selected and features that are not in the set are unselected. A configuration is valid if it
satisfies all the constraints of the feature model, otherwise, it is called invalid.
\citet{batory2005feature} shows, that a feature model can be defined as a propositional formula. This not only allows
the feature model to be stored in file formats like DIMACS, but it also allows the use of of-the-shelf SAT-Solvers for
feature models.


\end{document}