\documentclass[../../thesis.tex]{subfiles}
\graphicspath{{./resources/} }
\begin{document}



\section{Influence model}\label{sec:group_sampling:influence_model}

With the groups created in the previous section, we can create a model,
which predicts the performance of a given system.
For this, we measure the non-functional property we want to predict for each group of features.
\autoref{tab:group_sampling:feature_influence} shows an example of such measurements.
We only enable the features of one group at a time. The taken measurement then serves
as the influence value of the group itself. In \autoref{tab:group_sampling:feature_influence}
each round of groupings is separated by a horizontal line. If we look at the first round of
groupings, the influence value of $G_1$ is 3 and the features $F_1$ and $F_2$ are part of the group.

If in another measurement, the same configuration is chosen, we expect to see the influence value
of the group as the measurement. Since measuring all possible groups (configurations) is not
feasible, we want to determine the influence of single features to predict unseen configurations.
By definition, only a few features are influential, this lets us assume that the influence measured
in a grouping stems from only one or just a few features. In our example, this lets us assume that $F_1$
and $F_2$ have the influence of 3. With multiple different groupings, we can
correct or confirm this assumption. The average influence value the feature has is our
best approximation with the data available. If an influential feature is in a group,
the group influence is most likely significantly higher. With enough groupings of different
features, we can determine an influential parameter due to the fact, that the average
influence of this feature is higher than that of the rest. In our example, we can see that
the groups with $F_6$ assigned to them have, on average, a higher influence than the groups
not containing $F_6$. With the average influence of the features, we can create a model of the system.
We can use the formula for multiple linear regression as described by \citet{uyanik2013multiplelinear}.

\begin{equation}
    y=\beta_0 + \beta_1 x_1 + ... + \beta_n x_n + \varepsilon
    % y = f_0 + f_1 I_1 + f_2 I_2 + ... + f_n I_n + \varepsilon
\end{equation}

We adjust the formula so that our parameters $\beta_n$ represents the influence of a feature.
The feature influences we determined in \autoref{tab:group_sampling:feature_influence}
represent the measured non-functional property of the system and include its baseline performance.
This means, the performance of the system if all features would be disabled.
We need to compensate for this since it would otherwise make the prediction of the model unusable.
We can do this by determining the baseline performance and subtracting it from the influence values.
An approximation of the baseline performance would be the average of all measurements taken.
The model constructed would be described by following forumlas:

\begin{equation}
    f_0 = \frac{1}{n}\sum_{n=1}^{|I|}I_n
\end{equation}\label{eq:group_sampling:intercept}
\begin{equation}
    f_n = I_n - f_0
\end{equation}\label{eq:group_sampling:coefficient}
\begin{equation}
    y = f_0 + f_1 x_1 + f_2 x_2 + ... + f_n x_n + \varepsilon
\end{equation}


Where $I_n$ is the average of all measurements taken which included $F_n$ and $x_n$ is either one, if the feature is selected
or 0, if the feature is not selected.
We can use this formula to predict the behaviour of the system on unseen configurations, but the accuracy
of the prediction is most likely not very good. We can see why if we look at $F_3$ and $F_5$ and their
respective average measurements $I_3$ and $I_4$. They both have high average values, due to the fact, that they
share a group with our influential feature $F_6$. The effect on their influence values due to sharing
a group with an influential feature would average down if the number of groupings would be increased, but
it is still a problem. We try to compensate for this problem with a stepwise analysis of the influence values
in \autoref{sec:optimization:stepwise_influence}.

% We assume only a few features are influential \cite{akers1978binarydecisiondiagram} \todo{wo ist das paper?}.



\begin{table}[h]
    \caption{ Feature groupings and their influences }
    \begin{center}
        \begin{tabular}{ccc|c|cccccc|cccccc}\toprule
            $G_1$                 & $G_2$ & $G_3$ & $R$  & $F_1$ & $F_2$ & $F_3$         & $F_4$ & $F_5$ & $F_6$ & $I_1$ & $I_2$ & $I_3$ & $I_4$ & $I_5$ & $I_6$ \\ \midrule
            1                     & 0     & 0     & 3    & 1     & 1     & 0             & 0     & 0     & 0     & 3     & 3     &       &       &       &       \\
            0                     & 1     & 0     & 6    & 0     & 0     & 1             & 1     & 0     & 0     &       &       & 6     & 6     &       &       \\
            0                     & 0     & 1     & 28   & 0     & 0     & 0             & 0     & 1     & 1     &       &       &       &       & 28    & 28    \\ \midrule
            1                     & 0     & 0     & 2    & 1     & 0     & 0             & 1     & 0     & 0     & 2     &       &       & 2     &       &       \\
            0                     & 1     & 0     & 7    & 0     & 1     & 0             & 0     & 1     & 0     &       & 7     &       &       & 7     &       \\
            0                     & 0     & 1     & 25   & 0     & 0     & 1             & 0     & 0     & 1     &       &       & 25    &       &       & 25    \\ \midrule
            \multicolumn{10}{l}{} & 2,5   & 5     & 15,5 & 4     & 17,5  & \textbf{26,5}                                                                         \\ \midrule
        \end{tabular}
    \end{center}\label{tab:group_sampling:feature_influence}%
\end{table}



\end{document}


