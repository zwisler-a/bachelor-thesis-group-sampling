\documentclass[../../thesis.tex]{subfiles}
\graphicspath{{./resources/} }
\begin{document}


\section{Group sampling with configuration options}

In order to implement group sampling on configuration options, the method described by \Citeauthor{saltelli2008global}
needs to be adapted. For simplicity, we only look at feature models with binary features.
When grouping features, each group of features needs to be a valid configuration if all features
in a group are enabled. Otherwise, measuring the influence of the group on its own is not possible.
This forces us to adhere to the constraints on the feature model while creating groups.
We can identify three different types of features, which affect the way we can assign features to groups.

\subsubsection{Mutually exclusive features}
In this work, we call a group of features, where we can not enable more than one at the same time,
a group of mutually exclusive features and consequently a feature contained in one of those groups,
a mutually exclusive feature.
For example, an alternative group is always a group of mutually exclusive features,
since only one feature can be enabled without violating the constraints.
Mutually exclusive features are problematic when we group features.
We can't have two features assigned to the same group if they are together in a mutually exclusive group.
This would cause an invalid configuration once the features of the group are enabled.

\subsubsection{Independent features} 
We call features, which do not have any constraints on them and are optional, independent features.
If a set of features already are a valid configuration, these features can be added or removed
completely independent of the already selected features. The resulting set of features would
still be a valid configuration. These features allow for the easy creation of groups among them
since they can be combined in any way possible without violating any constraints.

\subsubsection{Implying feature}
Implying features, or as \Citeauthor{benavides2010automated} categorizes them, requires features \cite{benavides2010automated},
are features with a constraint, which forces us to select the implied feature
if the implying feature is enabled. If an implying feature is assigned to a group, we need to
assign the implied feature to the same group, otherwise, we would get an invalid configuration
if the group with the implying feature is selected and the one with the implied feature is not.

\subsubsection{}

In \citet{saltelli2008global} group sampling, each group is a distinct set of parameters.
This limits the number of groups possible on a set of features with constraints.
While mandatory features would make it impossible to create any groups, we simply could ignore them, since they do not impact the
performance of a given system. The most limiting factor would be a mandatory group of mutually exclusive features.
The number of possible groups with a distinct set of features would equal the smallest mandatory group of mutually exclusive features.
If a feature model contains multiple sets of mutually exclusive groups of different sizes, it is impossible
to assign all mutually exclusive features to a group while still having a distinct set of features for each group.

To create a performance influence model we need to measure the influence of each group individually.
By doing so, we can estimate the influence of each group of features. With repeated testing of different groupings,
we can estimate the influence of each feature by averaging the influence of the feature across all groupings.
While this does not give us a perfect estimate of the influence of each feature, it lets us test, which
of the features is most likely influential and to what degree.



\end{document}


