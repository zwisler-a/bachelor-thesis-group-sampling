
\documentclass[../thesis.tex]{subfiles}
\providecommand{\myfloor}[1]{\left \lfloor #1 \right \rfloor }
\graphicspath{{./resources/} }
\begin{document}


\chapter{Group Sampling}\label{chapter:group_sampling}


\subfile{03_group_sampling/03_1_introduction.tex}

\subfile{03_group_sampling/03_2_with_configurations.tex}

\subfile{03_group_sampling/03_3_group_creations.tex}

\subfile{03_group_sampling/03_4_influence_model.tex}


\subsection{Multicollinearity}

In regression analysis, having correlation among predictors is undesirable \cite{daoud2017multicollinearity}.
If two or more predictors in a multiple regression model have a linear relation, it is called multicollinearity.
Multicollinearity increases the standard errors and makes the coefficient unreliable, decreasing their precision \cite{alin2010multicollinearity}.
We can use the variance inflation factor (VIF) to find multicollinearity in our data.
\citet{alin2010multicollinearity} provides us with the formula:
\begin{equation}
    VIF_i = \frac{1}{1-R^{2}_i}\ for\ i=1,2,...,k
\end{equation}\label{eq:group_sampling:vif}

Where $ R^{2}_i $ is the coefficient of multiple determination of independent variable $x_i$ on the remaining variables \cite{alin2010multicollinearity}.
We can calculate the VIF by performing a regression on all independent variables except one, for each independent variable. 
Each created model gives us an $R^{2}$ value, which we can use in \autoref{eq:group_sampling:vif} to determine the VIF for
an independent variable. A VIF greater than 5 indicates a high correlation \cite{daoud2017multicollinearity}.



\subsection{Feature interactions}\label{sec:group_sampling:interactions}

With group sampling, we collect information about features grouped together.
This includes the interactions between features. To make use of this information,
we can add the interactions of features to our calculation of the influence of individual features.
A way to do this is to treat interactions similar to features during the determination
of feature influences. By assigning the group value to the interaction, if and only if all features
involved in the interaction are in the group, we can estimate the influence of the interaction 
the same way as with features. 
In \autoref{tab:group_sampling:interactions} we added each interaction $I_{i,j}$ between two features $F_i$ and $F_j$ to
the table.


\begin{table}[h]
    \caption{ Feature groupings and their influences }
    \begin{center}
        \begin{tabular}{cc|c|ccc|cccccc}\toprule
            $G_1$                & $G_2$ & $R$ & $F_1$ & $F_2$ & $F_3$ & $I_1$       & $I_2$ & $I_3$ & $I_{1,2}$ & $I_{2,3}$ & $I_{1,3}$ \\ \midrule
            1                    & 0     & 3   & 1     & 1     & 0     & 3           & 3     &       & 3         &           &           \\
            0                    & 1     & 6   & 0     & 0     & 1     &             &       & 6     &           &           &           \\ \midrule
            1                    & 0     & 19  & 1     & 0     & 1     & 19          &       & 19    &           &           & 19        \\
            0                    & 1     & 7   & 0     & 1     & 0     &             & 7     &       &           &           &           \\\midrule
            \multicolumn{6}{l}{} & 11    & 5   & 12,5  & 3     &       & \textbf{19}                                                     \\ \midrule
        \end{tabular}
    \end{center}\label{tab:group_sampling:interactions}%
\end{table}

$I_{1,2}$, $I_{2,3}$ and $I_{1,3}$ are the interactions between the features and get assigned the
group value if both features of the interaction are selected in the group. With a smaller group
size, we capture more interactions during our measurements since more features are selected in one group.
We can see in our example, that the interaction between $F_1$ and $F_3$ results in a higher measurement
as if the features are grouped with another feature or are in a group alone. 
While we were able to capture some interactions, we can see that we did not capture the interaction between
$F_2$ and $F_3$. To estimate the influence of all interactions we need a much larger sample size than to estimate 
the influence of features on their own.



\subfile{03_group_sampling/03_6_optimization.tex}


\end{document}