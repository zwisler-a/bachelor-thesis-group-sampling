\documentclass[../thesis.tex]{subfiles}
\externaldocument[I-]{./basics}
\graphicspath{{\subfix{../resources/}}}
\begin{document}

\definecolor{mygreen}{rgb}{0,0.6,0}
\definecolor{mygray}{rgb}{0.5,0.5,0.5}
\definecolor{mymauve}{rgb}{0.58,0,0.82}
\lstset{
    backgroundcolor=\color{white},   % choose the background color; you must add \usepackage{color} or \usepackage{xcolor}; should come as last argument
    basicstyle=\tiny,        % the size of the fonts that are used for the code
    breakatwhitespace=false,         % sets if automatic breaks should only happen at whitespace
    breaklines=false,                 % sets automatic line breaking
    captionpos=b,                    % sets the caption-position to bottom
    commentstyle=\color{mygreen},    % comment style
    deletekeywords={...},            % if you want to delete keywords from the given language
    escapeinside={\%*}{*)},          % if you want to add LaTeX within your code
    extendedchars=true,              % lets you use non-ASCII characters; for 8-bits encodings only, does not work with UTF-8
    firstnumber=1,                % start line enumeration with line 1000
    frame=single,	                   % adds a frame around the code
    keepspaces=true,                 % keeps spaces in text, useful for keeping indentation of code (possibly needs columns=flexible)
    keywordstyle=\color{blue},       % keyword style
    language=Python,                 % the language of the code
    morekeywords={*,...},            % if you want to add more keywords to the set
    numbers=left,                    % where to put the line-numbers; possible values are (none, left, right)
    numbersep=5pt,                   % how far the line-numbers are from the code
    numberstyle=\tiny\color{mygray}, % the style that is used for the line-numbers
    rulecolor=\color{black},         % if not set, the frame-color may be changed on line-breaks within not-black text (e.g. comments (green here))
    showspaces=false,                % show spaces everywhere adding particular underscores; it overrides 'showstringspaces'
    showstringspaces=false,          % underline spaces within strings only
    showtabs=false,                  % show tabs within strings adding particular underscores
    stepnumber=1,                    % the step between two line-numbers. If it's 1, each line will be numbered
    stringstyle=\color{mymauve},     % string literal style
    tabsize=2,	                   % sets default tabsize to 2 spaces
}


\appendix
\chapter{Implementation}

In the following section, the important parts of our implementation of
the group sampling algorithm are included. The Implementation is written 
in python and uses common packages out of its ecosystem. Most notably 
\textbf{sklearn} for regression methods, \textbf{pandas} and \textbf{numpy} 
for data manipulation, \textbf{networkx} for graph related tasks and \textbf{z3-solver}
as a wrapper for python to use the Z3 Theorem prover.

\section{Sampling Stratgies}
\renewcommand{\lstlistingname}{Strategy}
\lstinputlisting[language=Python,
    frame=single,
    caption={Hamming distance based group sampling stratgey},
    captionpos=b]
{chapters/99_apendix/hamming_group_sampling.py}


\lstinputlisting[language=Python,
    frame=single,
    caption={Group sampling stratgey with independent features},
    captionpos=b]
{chapters/99_apendix/independent_group_sampling.py}


\section{Graph algorithms}
\renewcommand{\lstlistingname}{Algorithm}
\lstinputlisting[language=Python,
    frame=single,
    caption={Generation of the mutually exclusive feature graph},
    captionpos=b]
{chapters/99_apendix/generate_mutex_graph.py}

\lstinputlisting[language=Python,
    frame=single,
    caption={Find true optional features},
    captionpos=b]
{chapters/99_apendix/optional_features.py}


\section{Group Sampling - Learning}
\lstinputlisting[language=Python,
    frame=single,
    caption={Generate model},
    captionpos=b]
{chapters/99_apendix/berta.py}



\end{document}