\documentclass[../thesis.tex]{subfiles}
\graphicspath{{\subfix{../resources/}}}
\begin{document}

\chapter{Conclusion and future work}
\section{Conclusions}
In this work, we looked at how one can create a performance influence model using
a group sampling approach. To do this, we devised two strategies to create groups
among configuration options and their constraints with the help of an SAT-solver. 
We identified mutually exclusive features as the most limiting factor during group
creation and allowed for mutually exclusive features to be in multiple groups to
circumvent the issue. With the groups, we determined the influence of single
features using the approach described by \Citeauthor{saltelli2008global} and
created a performance influence model. This model was then evaluated on
multiple datasets against a random sampling approach with linear regression.
We also looked at how one could include feature interactions in the resulting 
performance influence model.

In our tests, we found that the group sampling approach while being able to identify
influential features, did not result in a reliable model. We argue that this is due
to the fact that the approach chosen to create the feature model is not able to accurately 
determine the influence of non-influential features. 
Leading to worse performance of the model on feature models with more features.
With the inclusion of feature interactions, we were able to identify some interactions,
if they appeared in the samples, but the model also proved to be unreliable due to the approach chosen.
Since our approach treats interactions and features similar when determining the influences,
the model also suffered from high average errors on non-influential features and interactions.

Even though group sampling was not able to make accurate predictions about a software system,
it was able to identify influential features with only a few measurements. 


\section{Future work}

Group sampling proves to be able to identify influential features with just a few measurements.
With a round-based approach, this identification might even be possible with fewer measurements.
By evaluating the influences after each round of groupings, one could create the following
groups in a way to more easily identify an influential feature. 
Another interesting approach to perform sensitivity analysis on a feature model would be 
to apply the elementary effects method on groups of features maybe allowing for an efficient
identification of influential features and interactions between features.

\end{document}